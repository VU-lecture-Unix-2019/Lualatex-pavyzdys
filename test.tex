\documentclass{article}
\usepackage{amsmath}
\begin{document}

\author{A.\ Utorius}
\title{Lietuvi{\v s}kų raid{\v z}ių ra{\v s}ymas}
\date{\today}

\maketitle

Lietuviškos raid{\.e}s:$<$ĄČĘĖĮŠŲŪŽąčęėįšųūž$>$ \cite{ref1}.

\section{Apžvalga}

Viena svarbiausių analizinės algebros žinių yra Teiloro eilutė:
\begin{align}
f(x)\big|_{x=a} &=& f(a)
 + \frac{f'(a)}{1!} (x-a)
 + \frac{f''(a)}{2!} (x-a)^2
 + \frac{f'''(a)}{3!} (x-a)^3
 + \dots \nonumber \\
 &=& \sum\limits_{n=0}^{\infty} \frac{f^{(n)}(a)}{n!} (x-a)^n
\end{align}

% ---------------------------------------------
\section*{Padėka}

\textit{\input{greetings}}

\begin{thebibliography}{99}
\bibitem{ref1} A.\ Bėcėlė (unpublished).
\end{thebibliography}
\end{document}
